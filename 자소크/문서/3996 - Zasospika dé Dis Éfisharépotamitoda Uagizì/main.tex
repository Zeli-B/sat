\documentclass{article}

\usepackage{kotex}

\title{
	\textsc{Zasospika dé Dis \\ Éfisharépotamitoda Uagizì \\
	\large 현대 자소크어의 공식 표기법 변경}
}
\author{Skara Simett Modelai Pracir XLII @ Enifia dé Spikédimerda}
\date{3996. 21. 2}

\begin{document}

\maketitle

\pagebreak

\section{개요}

현대에 오면서 자소크어의 발음이 상당히 달라지기 시작했다. 발음이 어려운 합성어는 과감히 축약하고, 축약은 문장에서 단어 사이에 연음 현상까지로 이어졌다. 언어표준청에서는 이러한 자소크어의 변화에 따라 자소크의 공식 라틴 문자 정서법을 개정하려고 한다.

이 문서는 해당 정서법 개정에 대한 내용을 담고 있으며, 앞으로는 이 문서에서 정한 정서법에 따라서 자소크어를 표기해야 할 것이다.

\section{내용}

\subsection{É 연음}

명사구 안에서의 é는 a나 i 등으로 이어질 때에 발음이 이어지지 못하고 끊어지는 현상이 발생다. 이에 자소크어는 éa를 [야], éo를 [요], éi를 [의], éy를 [이]로 발음하는 방식으로 변화가 일어났다. 각각의 변화에 따라 다음 정서법을 적용한다.

\begin{itemize}
	\item éa[야] $\rightarrow$ á. Éa는 [ä]와 같은 음가를 가지지 않는다. 굳이 적자면 [ia]와 같은 음가를 지닌다.
		\begin{itemize}
			\item mir\underline{éa}dor $\rightarrow$ mir\underline{á}dol \\ 나의 도구
			\item his\underline{éa}st $\rightarrow$ his\underline{á}st \\ 그 선택
		\end{itemize}
	\item éo[요] $\rightarrow$ ó.
		\begin{itemize}
			\item insürta\underline{éo}liatcéski $\rightarrow$ insürt\underline{ó}gliacéski \\ 진짜 오글거리는 것
		\end{itemize}
	\item éi[의] $\rightarrow$ í. 영어의 ``Yee"와 비슷한 소리가 난다.
		\begin{itemize}
			\item insürta\underline{éi}nsürta $\rightarrow$ insürt\underline{í}nsürta \\ 진짜 진실
		\end{itemize}
	\item éy[의] $\rightarrow$ ý[i].
		\begin{itemize}
			\item Zasoquspika\underline{éy}úrgos $\rightarrow$ Zasospika\underline{ý}úrgos \\ 자소크어의 단어들
		\end{itemize}
\end{itemize}

\subsection{합성어 연음}

합성어가 만들어지고서, 단어와 단어 사이에 발음이 이어지지 못하고 끊어지는 현상이 발생한다. 이에 자소크어는 이어지는 두 음소 중 뒤쪽에 있는 음소를 살려서 발음하게 되었다. 이를 표기에 반영한다.

\begin{itemize}
	\item 자소크어; Zaso\underline{qus}pika → Zaso\underline{s}pika\footnote{따라서 자소크어의 공식 명칭은 변경된다.}
	\item 정부; nati\underline{oa}centem → nati\underline{a}centem
	\item 버릇; nia\underline{shd}aterma → nia\underline{d}aterma
	\item 협력; karo\underline{ps}ki → karo\underline{s}ki
	\item 챙겨주다; koni\underline{rv}ist → koni\underline{v}ist
	\item 주어; acente\underline{mn}amyúrgo → acente\underline{n}amyúrgo
	\item 열매; aiki\underline{ff}udi → aiki\underline{f}udi
\end{itemize}

단, 뒤쪽에 있는 단어의 초성이 h인 경우에는 이를 생략한다.

\begin{itemize}
	\item 소방차; fira\underline{dh}ashalin → fira\underline{d}ashalin
	\item 직선; \underline{yh}ülienski → \underline{y}ülienski
	\item 달빛; faru\underline{mh}inoc → faru\underline{m}inoc
\end{itemize}

단, 다음에 해당되는 규칙은 예외로써 적용한다.

\begin{itemize}
	\item shs, shz는 sh로 발음하고 표기한다.
		\begin{itemize}
			\item 자연어; ve\underline{shs}pika → ve\underline{sh}pika
			\item 태어났을 때; inatanti\underline{shz}ozaque → inatanti\underline{sh}ozaque
		\end{itemize}
	\item fv는 f로 발음하고 표기한다.
		\begin{itemize}
			\item 만들어 보다; kri\underline{fv}izash → kri\underline{f}izash
		\end{itemize}
	\item tz, ts, cz는 c로 발음하고 표기한다.
		\begin{itemize}
			\item 생시(生時); do\underline{ts}osaque → do\underline{c}osaque
		\end{itemize}
\end{itemize}

단, 유음화가 일어나지 않는 곳에서 비음화가 일어난다면 이를 표기와 발음에 반영한다.

\begin{itemize}
	\item 합창; karopmaque → karommaque
\end{itemize}
\begin{enumerate}
	\item 단, nv는 mv로 표기하고 발음한다.
		\begin{itemize}
			\item 이온화; ionvesh → iomvesh
		\end{itemize}
	\item 비음화가 일어난 `nn`은 `nxn`으로 표기한다.
\end{enumerate}

\subsection{명사-조사 단어 연음}

합성어가 만들어지고서, 단어와 단어 사이에 발음이 이어지지 못하고 끊어지는 현상이 발생한다. 이에 자소크어는 이어지는 두 음소 중 뒤쪽에 있는 음소를 살려서 발음하게 되었다. 이를 표기에 반영한다.

\begin{itemize}
	\item Uat 'egro\footnote{Uat degro[uategro] $\rightarrow$ Uat 'egro} zur izùll zur degro uatý izùm. \\ 말도 행동이고, 행동도 말의 일종이다.
\end{itemize}

\subsection{일반적인 잔음의 표기}

단어와 단어 사이에서 일어나는 것이 아니라 일반적으로 어말에서 ``을르"와 같이 소리나는 것을 잔음(ftidino)이라고 하고, 이를 표기할 때에는 잔음의 음소를 2번 작성한다.

\begin{itemize}
	\item 노래를 불러서; maqû\underline{le} $\rightarrow$ maqû\underline{ll}
\end{itemize}

\end{document}
