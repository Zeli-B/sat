\documentclass{article}

\usepackage{kotex}

\newcounter{question}
\newenvironment{question}[1][]{
    \refstepcounter{question}
    \textbf{문제~\Roman{question}.} #1
}{\medskip}

\title{\textbf{철학}}
\author{3673년도 에빈카바스 시험지}
\date{}

\begin{document}

\maketitle

당신은 에빈카바스의 마법 시험을 보고 있다. 에빈카바스의 마법 시험은 모든 자소크의 시험 중 가장 어렵다고 알려졌는데, 이는 마법을 부리기 위한 최대한의 체력과 마법적 지식, 그 지식들을 어우르는 지혜와, 오른 시간 두뇌 사용을 유지하는 지구력 등을 고루 평가하기 때문이라고 한다. 이때, 다음 상황에서 당신이 할 생각과 행동을 서술하시오.
\medskip

\begin{question}
    답을 도출하는 방법을 알고 있었으나 그 방법이 생각나지 않는다. 하지만 당신은 방법을 도출할 통찰력이 있으나, 시간이 턱없이 부족하다. 이 문제를 해결할 것인가, 풀 수 있을 다른 문제를 찾아볼 것인가?
\end{question}

\begin{question}
    시험 시간이 얼마 남지 않았다. 하지만 남은 시간에 비해 완료해야 할 과제가 턱없이 많다. 이때, 감독관이 큰 소리로 “10분 남았습니다. 이제 정리해주세요.”라는 말을 해, 집중이 잘 되지 않아, 풀 수 있던 문제도 못 풀게 생겼다. 원래 시험감독관이 그러한 행동을 하는 것으로 정해져있지 않았을 때, 당신은 시험감독관을 어떻게 판단할 것인가?
\end{question}

\begin{question}
    당신은 땅에 떨어져있는 종이를 발견했다. 그 내용을 확인해보니 이번 시험의 정답을 써놓은 에빈카바스의 공식 문서였다. 당신이 이 답안지를 확인하고 그대로 진행한다면, 당신은 에빈카바스에서 마법 부문 만점을 받고 원하는 인생을 살 수 있다. 당신이 이 답안지를 모른척한다면, 시험의 감독관은 감독능력 부족으로 감독관은 해고당하게 되며, 해당 시험은 무효처리된다. 당신이 감독과에게 이 사실을 조용히 알린다면 감독관은 이 일을 당신과 자신만의 비밀로 둘 것이고, 시험은 평소처럼 진행될 것이다. 지금까지 누가 그 답안지를 고의적으로 무시했는지, 당장은 알 수가 없다. 당신은 어떤 행동을 보일 것인가?
\end{question}

모든 항목은 납득할 수 있는 근거와 사상을 들어 서술하시오.

\begin{center}
| 수고하셨습니다. 문제지를 체줄하시고 퇴실하셔도 좋습니다. |
\end{center}

\end{document}