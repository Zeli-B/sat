\documentclass{article}

\newcounter{question}
\newenvironment{question}[1][]{
    \refstepcounter{question}
    \textbf{Vlazif~\Roman{question}}. #1
}{\medskip}

\title{\textbf{Erasheniluo}}
\author{3151 Ärge Y'Evinkavas Évlazifcamis}
\date{}


\begin{document}

\maketitle

\begin{question}
    Steli'l dürziaàsùle, hisé marsh lo vizashàsòm.
\end{question}

\begin{question}
    ``Miro fas siro?'' u narisàsòm.
\end{question}

\begin{question}
    Doéquorsin \textit{y'Avnires}'d hanöskrifèv émaka; \textit{y'Holpotela} da,
    has da tagèvézozak, grosgesdaixizùm ad
    mashaéshain lo vagùh tou tagùm tia,
    diornaquo'd his u kavùle shainé zur lo vizashùkxnahùm ha
    Avnires fas diornaquo é \textit{ë Vanisha} gi
    hasétag lo darutùi lofanùkùm é \textit{ë Yúprenad}xèv.
    Rifo Vanisha fas Holpotela da
    grosgesdaixizùle mashaéshain lo vagùlexkomàdéski touatùi tagùle
    Avnires da Holpotelaxùméski lo ürpatxtrendèv.
    Tia Vanisha fas Avnires samish diornaùi
    Avnires ad Vanisha é Holpotela u kav éhazidiornaski'l narisùle
    ä hazidiornaski'd Holpotela lo grosgesdaiùi vizashùkàs tou tagèv.
    Rifo Vanisha fas Avnires rad turadùi Yúprenadxèvxrifo,
    Avnires fas Yúprenad da Holpotela é nuiz tou tagèvùle
    Vanisha'd hasxsamishékaroptrend lo aifidomùmé-ski tou tagèv.
    Rifo Avnires fas Vanisha u Yúprenadxàsèvxtia,
    Vanisha fas Yúprenad da Holpotela é nuiz tou tagxnahèvùle Avnireséomeski'l saliquèv.
    Avnires fas Vanisha'd Holpotelaxàdéski tou tagùle
    Vanisha'd Holpotelaxùméski u kavùmé trendé \textit{ë Pleva} l'ol krifùle konirèvxtia,
    Vanisha'd Yúprenadxnahàdé-ski'l narisùle Pleva lo sazigauùnèv.
    Hisézozak, Avnires gifas Holpotelaxùméski u kavùle
    insürtaiksisépuat é \textit{ë Zevrodes} da krifùkèv.
    Dis u kavùle Avnires fas Vanisha gi maneyàsèv.

    Insürtaùi Holpotela fas Yúprenad lo apelùle veshùm touatùm ézozak,
    Avnires ad Vanisha é zur u kavùm é hasétag lo vizashùkùle,
    sihémarsh lo hülien az Avnires ad Vanisha acentem svo
    Zevrodes u kavùle oh sulah émaney lo vagìc équorsin lo vizashùkòm.
\end{question}

\begin{center}
    | Lomiaèv. Vlazifcamis lo tempukùle quinveshùhùm. |
\end{center}

\end{document}
