\documentclass{article}

\usepackage{kotex}

\newcounter{question}
\newenvironment{question}[1][]{
    \refstepcounter{question}
    \textbf{문제~\Roman{question}.} #1
}{\medskip}

\title{\textbf{철학}}
\author{3151년도 에빈카바스 시험지}
\date{}


\begin{document}

\maketitle

\begin{question}
    행복을 정의하고, 그것의 이유를 밝히시오.
\end{question}

\begin{question}
    ``나는 누구인가?''를 답하시오.
\end{question}

\begin{question}
    어떤 사람 \textit{아브니레스}는 자신이 창업한 일 \textit{올포텔라}가
    자신이 생각했을 때에 가망이 있고
    많은 이로움을 가질 수 있으리라 기대하지만,
    다른 사람이 그에 대해 적당한 반응을 보여주지 않자
    다른 사람 \textit{바니샤}에게 자신의 생각을 직접 알리는 \textit{여프레나드}를 했다.
    이에 바니샤는 올포텔라가 바람직하고 많은 이로움을 가지고 올 것이라는 판단 하에
    아브니레스가 올포텔라하는 것을 돕기로 했다.
    하지만 바니샤는 아브니레스와 달리 아브니레스와 바니샤의 올포텔라에 대한 특이성을 알고,
    그 특이성이 올포텔라를 가망있게 보인다고 생각했다.
    따라서 바니샤는 아브니레스보다 소극적으로 여프레나드했는데,
    아브니레스는 여프레나드가 올포텔라의 일부라고 생각했기 때문에 바니샤가
    자신과의 약속을 어긴 것으로 생각했다.
    따라서 아브니레스는 바니샤에게 여프레나드하라고 요청했으나,
    바니샤는 여프레나드가 올포텔라의 일부라고 생각하지 않았기 때문에 아브니레스의 요청을 거부했다.
    아브니레스는 바니샤가 올포텔라할 것이라고 기대하고
    바니샤가 올포텔라하는 일에 대한 계획 \textit{플레바}를 모두 수립하여 두었으나,
    바니샤가 여프레나드하지 않을 것이라는 것을 알고 플레바를 철회해야 했다.
    이때 아브니레스에게는 올포텔라하는 것에 대해 실질적인 저항 \textit{제브로데스}가 발생했다.
    이에 대해 아브니레스는 바니샤에게 책임을 요구하였다.

    실제로 올포텔라는 여프레나드를 통해 이루어진다고 할 때,
    아브니레스와 바니샤의 행동에 대한 자신의 가치관을 밝히고,
    해당 가치관을 바탕으로 아브니레스와 바니샤 중 제브로데스에 대해
    더 큰 책임을 가진 사람을 밝히시오.
\end{question}

\begin{center}
    | 수고하셨습니다. 문제지를 제출하시고 퇴실하셔도 좋습니다. |
\end{center}

\end{document}
